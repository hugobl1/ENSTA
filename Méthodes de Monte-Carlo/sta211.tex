% Options for packages loaded elsewhere
\PassOptionsToPackage{unicode}{hyperref}
\PassOptionsToPackage{hyphens}{url}
%
\documentclass[
]{article}
\usepackage{lmodern}
\usepackage{amssymb,amsmath}
\usepackage{ifxetex,ifluatex}
\ifnum 0\ifxetex 1\fi\ifluatex 1\fi=0 % if pdftex
  \usepackage[T1]{fontenc}
  \usepackage[utf8]{inputenc}
  \usepackage{textcomp} % provide euro and other symbols
\else % if luatex or xetex
  \usepackage{unicode-math}
  \defaultfontfeatures{Scale=MatchLowercase}
  \defaultfontfeatures[\rmfamily]{Ligatures=TeX,Scale=1}
\fi
% Use upquote if available, for straight quotes in verbatim environments
\IfFileExists{upquote.sty}{\usepackage{upquote}}{}
\IfFileExists{microtype.sty}{% use microtype if available
  \usepackage[]{microtype}
  \UseMicrotypeSet[protrusion]{basicmath} % disable protrusion for tt fonts
}{}
\makeatletter
\@ifundefined{KOMAClassName}{% if non-KOMA class
  \IfFileExists{parskip.sty}{%
    \usepackage{parskip}
  }{% else
    \setlength{\parindent}{0pt}
    \setlength{\parskip}{6pt plus 2pt minus 1pt}}
}{% if KOMA class
  \KOMAoptions{parskip=half}}
\makeatother
\usepackage{xcolor}
\IfFileExists{xurl.sty}{\usepackage{xurl}}{} % add URL line breaks if available
\IfFileExists{bookmark.sty}{\usepackage{bookmark}}{\usepackage{hyperref}}
\hypersetup{
  pdftitle={{[}STA211{]} Ajustement d'une loi de Weibull sur des données de durée de vie d'un composant industriel, avec censures à droite},
  hidelinks,
  pdfcreator={LaTeX via pandoc}}
\urlstyle{same} % disable monospaced font for URLs
\usepackage[margin=1in]{geometry}
\usepackage{color}
\usepackage{fancyvrb}
\newcommand{\VerbBar}{|}
\newcommand{\VERB}{\Verb[commandchars=\\\{\}]}
\DefineVerbatimEnvironment{Highlighting}{Verbatim}{commandchars=\\\{\}}
% Add ',fontsize=\small' for more characters per line
\usepackage{framed}
\definecolor{shadecolor}{RGB}{248,248,248}
\newenvironment{Shaded}{\begin{snugshade}}{\end{snugshade}}
\newcommand{\AlertTok}[1]{\textcolor[rgb]{0.94,0.16,0.16}{#1}}
\newcommand{\AnnotationTok}[1]{\textcolor[rgb]{0.56,0.35,0.01}{\textbf{\textit{#1}}}}
\newcommand{\AttributeTok}[1]{\textcolor[rgb]{0.77,0.63,0.00}{#1}}
\newcommand{\BaseNTok}[1]{\textcolor[rgb]{0.00,0.00,0.81}{#1}}
\newcommand{\BuiltInTok}[1]{#1}
\newcommand{\CharTok}[1]{\textcolor[rgb]{0.31,0.60,0.02}{#1}}
\newcommand{\CommentTok}[1]{\textcolor[rgb]{0.56,0.35,0.01}{\textit{#1}}}
\newcommand{\CommentVarTok}[1]{\textcolor[rgb]{0.56,0.35,0.01}{\textbf{\textit{#1}}}}
\newcommand{\ConstantTok}[1]{\textcolor[rgb]{0.00,0.00,0.00}{#1}}
\newcommand{\ControlFlowTok}[1]{\textcolor[rgb]{0.13,0.29,0.53}{\textbf{#1}}}
\newcommand{\DataTypeTok}[1]{\textcolor[rgb]{0.13,0.29,0.53}{#1}}
\newcommand{\DecValTok}[1]{\textcolor[rgb]{0.00,0.00,0.81}{#1}}
\newcommand{\DocumentationTok}[1]{\textcolor[rgb]{0.56,0.35,0.01}{\textbf{\textit{#1}}}}
\newcommand{\ErrorTok}[1]{\textcolor[rgb]{0.64,0.00,0.00}{\textbf{#1}}}
\newcommand{\ExtensionTok}[1]{#1}
\newcommand{\FloatTok}[1]{\textcolor[rgb]{0.00,0.00,0.81}{#1}}
\newcommand{\FunctionTok}[1]{\textcolor[rgb]{0.00,0.00,0.00}{#1}}
\newcommand{\ImportTok}[1]{#1}
\newcommand{\InformationTok}[1]{\textcolor[rgb]{0.56,0.35,0.01}{\textbf{\textit{#1}}}}
\newcommand{\KeywordTok}[1]{\textcolor[rgb]{0.13,0.29,0.53}{\textbf{#1}}}
\newcommand{\NormalTok}[1]{#1}
\newcommand{\OperatorTok}[1]{\textcolor[rgb]{0.81,0.36,0.00}{\textbf{#1}}}
\newcommand{\OtherTok}[1]{\textcolor[rgb]{0.56,0.35,0.01}{#1}}
\newcommand{\PreprocessorTok}[1]{\textcolor[rgb]{0.56,0.35,0.01}{\textit{#1}}}
\newcommand{\RegionMarkerTok}[1]{#1}
\newcommand{\SpecialCharTok}[1]{\textcolor[rgb]{0.00,0.00,0.00}{#1}}
\newcommand{\SpecialStringTok}[1]{\textcolor[rgb]{0.31,0.60,0.02}{#1}}
\newcommand{\StringTok}[1]{\textcolor[rgb]{0.31,0.60,0.02}{#1}}
\newcommand{\VariableTok}[1]{\textcolor[rgb]{0.00,0.00,0.00}{#1}}
\newcommand{\VerbatimStringTok}[1]{\textcolor[rgb]{0.31,0.60,0.02}{#1}}
\newcommand{\WarningTok}[1]{\textcolor[rgb]{0.56,0.35,0.01}{\textbf{\textit{#1}}}}
\usepackage{graphicx,grffile}
\makeatletter
\def\maxwidth{\ifdim\Gin@nat@width>\linewidth\linewidth\else\Gin@nat@width\fi}
\def\maxheight{\ifdim\Gin@nat@height>\textheight\textheight\else\Gin@nat@height\fi}
\makeatother
% Scale images if necessary, so that they will not overflow the page
% margins by default, and it is still possible to overwrite the defaults
% using explicit options in \includegraphics[width, height, ...]{}
\setkeys{Gin}{width=\maxwidth,height=\maxheight,keepaspectratio}
% Set default figure placement to htbp
\makeatletter
\def\fps@figure{htbp}
\makeatother
\setlength{\emergencystretch}{3em} % prevent overfull lines
\providecommand{\tightlist}{%
  \setlength{\itemsep}{0pt}\setlength{\parskip}{0pt}}
\setcounter{secnumdepth}{-\maxdimen} % remove section numbering

\title{{[}STA211{]} Ajustement d'une loi de Weibull sur des données de durée de
vie d'un composant industriel, avec censures à droite}
\author{}
\date{\vspace{-2.5em}}

\begin{document}
\maketitle

\textbf{\emph{Simulation:}}

\textbf{Question 1:}

Calculons la fonction \(F^{-1}(p|\alpha,\kappa)\):

Soit \(y = F(x) = 1 - exp(-\alpha x^{\kappa})\),

Alors on a: \begin{align}
 exp(-\alpha x^{\kappa})  &= 1 - y \\
\Rightarrow  x^{\kappa} &= -\dfrac{\log(1-y)}{\alpha} \\
 \Rightarrow  x &= \left(-\dfrac{\log(1-y)}{\alpha}\right)^{1/\kappa} 
\end{align}

Ainsi:

\(F^{-1}(p|\alpha,\kappa)=\left(-\dfrac{\log(1-y)}{\alpha}\right)^{1/\kappa}\)

Nous allons maintenant utiliser la méthode d'inversion générique qui
nous donne le résultat suivant:

Soit \(U\sim\mathcal U(0,1)\), \(X=F^{-1}(U)\)

Alors, X est distribuée selon F: \[X=F^{-1}(U)\sim F \] Aussi, en
appliquant ce résultat à notre exercice on a:
\[X\sim\mathcal W(\alpha,\kappa) \] En prenant simplement:
\[X=F^{-1}(U|\alpha,\kappa)\] où \(U\sim\mathcal U(0,1)\).

On peut donc en déduire un algorithme de simulation de cette loi basée
sur l'inversion générique, on génère les réalisations d'une loi uniforme
\(\mathcal U(0,1)\) auxquelles on applique simplement la fonction
\(F^{-1}(u|\alpha,\kappa)\), ce qui permet d'obtenir des réalisations
simulées selon la loi de Weibull.

On programme en R, une fonction permettant de réaliser cette simulation
et qui nous renvoie n observations issue d'une loi de Weibull de
paramètres \(\alpha\) et \(\kappa\).

\begin{Shaded}
\begin{Highlighting}[]
\NormalTok{simul_weibull <-}\StringTok{ }\ControlFlowTok{function}\NormalTok{(n,alpha,kappa)}
\NormalTok{\{}
\NormalTok{u =}\StringTok{ }\KeywordTok{runif}\NormalTok{(n,}\DecValTok{0}\NormalTok{,}\DecValTok{1}\NormalTok{) ; }\CommentTok{#simulation d'une uniforme [0,1]}
\NormalTok{x =}\StringTok{ }\NormalTok{(}\OperatorTok{-}\KeywordTok{log}\NormalTok{(}\DecValTok{1}\OperatorTok{-}\NormalTok{u)}\OperatorTok{/}\NormalTok{alpha)}\OperatorTok{^}\NormalTok{\{}\DecValTok{1}\OperatorTok{/}\NormalTok{kappa\}}
\KeywordTok{return}\NormalTok{ (x)}
\NormalTok{\}}
\end{Highlighting}
\end{Shaded}

On vérifie cette fonction en comparant ces observations avec celles
issues de la fonction \emph{rweibull}, on obtient les figures suivantes
pour \(n=10000\), \(\alpha=1\) et \(\kappa=2\):

\begin{Shaded}
\begin{Highlighting}[]
\KeywordTok{hist}\NormalTok{(}\KeywordTok{simul_weibull}\NormalTok{(}\DecValTok{10000}\NormalTok{,}\DecValTok{1}\NormalTok{,}\DecValTok{2}\NormalTok{))}
\end{Highlighting}
\end{Shaded}

\includegraphics{sta211_files/figure-latex/unnamed-chunk-2-1.pdf}

\begin{Shaded}
\begin{Highlighting}[]
\KeywordTok{hist}\NormalTok{(}\KeywordTok{rweibull}\NormalTok{(}\DecValTok{10000}\NormalTok{,}\DecValTok{2}\NormalTok{,}\DecValTok{1}\NormalTok{))}
\end{Highlighting}
\end{Shaded}

\includegraphics{sta211_files/figure-latex/unnamed-chunk-2-2.pdf}

Notre fonction semble fonctionner correctement après comparaison des
histogrammes.

\textbf{Question 2:} Dans cette question, nous réalisons une fonction
simulant \(n\) observations de la loi de Weibull de paramètres
\(\alpha\) et \(\kappa\) donnés, censurée au-dessus d'un niveau \(t_0\)
donné, la fonction renverra le vecteur des données simulées ainsi qu'un
nombre \(p\) correspondant au nombre d'observations non censurées
(triées au début du vecteur).

\begin{Shaded}
\begin{Highlighting}[]
\NormalTok{simul_weibull_censur <-}\StringTok{ }\ControlFlowTok{function}\NormalTok{(n,alpha,kappa,t0)}
\NormalTok{\{}
\NormalTok{  x =}\StringTok{ }\KeywordTok{simul_weibull}\NormalTok{(n,alpha,kappa)}
\NormalTok{  x=}\KeywordTok{sort}\NormalTok{(}\KeywordTok{as.numeric}\NormalTok{(}\KeywordTok{lapply}\NormalTok{(x,}\ControlFlowTok{function}\NormalTok{(xi)\{}\KeywordTok{ifelse}\NormalTok{(xi}\OperatorTok{<=}\NormalTok{t0,xi,t0)\})))}
\NormalTok{  p=}\KeywordTok{length}\NormalTok{(}\KeywordTok{which}\NormalTok{(x}\OperatorTok{<}\NormalTok{t0))}
  \KeywordTok{return}\NormalTok{ (}\KeywordTok{list}\NormalTok{(x,p))}
\NormalTok{\}}

\KeywordTok{hist}\NormalTok{(}\KeywordTok{simul_weibull_censur}\NormalTok{(}\DecValTok{10000}\NormalTok{,}\DecValTok{1}\NormalTok{,}\DecValTok{2}\NormalTok{,}\DecValTok{1}\NormalTok{)[[}\DecValTok{1}\NormalTok{]], }\DataTypeTok{main=}\StringTok{"Loi de Weibull censurée"}\NormalTok{, }\DataTypeTok{xlab=}\StringTok{"Données simulées"}\NormalTok{)}
\end{Highlighting}
\end{Shaded}

\includegraphics{sta211_files/figure-latex/Question 2-1.pdf}

\begin{Shaded}
\begin{Highlighting}[]
\KeywordTok{hist}\NormalTok{(}\KeywordTok{simul_weibull_censur}\NormalTok{(}\DecValTok{10000}\NormalTok{,}\DecValTok{1}\NormalTok{,}\DecValTok{2}\NormalTok{,}\DecValTok{2}\NormalTok{)[[}\DecValTok{1}\NormalTok{]], }\DataTypeTok{main=}\StringTok{"Loi de Weibull censurée"}\NormalTok{, }\DataTypeTok{xlab=}\StringTok{"Données simulées"}\NormalTok{)}
\end{Highlighting}
\end{Shaded}

\includegraphics{sta211_files/figure-latex/Question 2-2.pdf}

Les histogrammes semblent cohérent, en effet on observe dans les deux
cas les observations censurées, par une augmentation du nombres de
réalisations valant 1 sur la première figure (car censure à 1) et d'une
augmentation du nombres de réalisations valant 2 sur la seconde figure
(car censure à 2) par rapport à la loi de Weibull calculées
précédemment.

\#Calcul de la vraisemblance:

\textbf{Question 3:}

La densité de notre variable aléatoire est donnée par :
\[f(t) = \alpha \kappa t^{\kappa-1} exp(-\alpha t^{\kappa})\]

La log-vraisemblance d'une observation non censurée \(t_i\) est
simplement donnée par la formule:
\[log(t_i,\alpha,\kappa,p) = log(\alpha) + log(\kappa) + (\kappa-1)log(t) - \alpha t^{\kappa}\]

Pour une censure à droite, la log-vraisemblance est égale à la fonction
de queue de la loi de Weibull :

\[log(t_i,\alpha,\kappa,p) = -\alpha t_{0}^{\kappa}\]

D'où, \begin{align}
l(t_{1:n}| \alpha,\kappa) &= \sum_{i = 1}^{p} log(t_i,\alpha,\kappa,p) +  \sum_{i = p+1}^{n}log(t_i,\alpha,\kappa,p)\\

 &= \sum_{i = 1}^{p} (log(\alpha) + log(\kappa) + (\kappa-1)log(t_i) - \alpha t_{i}^{\kappa}) +\sum_{i = p+1}^{n}-\alpha t_{0}^{\kappa}\\

 &= p(log(\alpha) + log(\kappa)) +\sum_{i = 1}^{p} (\kappa-1)log(t_i) -\sum_{i = 1}^{p} \alpha t_{i}^{\kappa}  - (n-p)\alpha t_{0}^{\kappa}\\
\end{align}

Donc, la log-vraisemblance dans le modèle de Weibull dont les \(n-p\)
premières données sont censurées à droites, s'écrit:

\[l(t_{1:n}| \alpha,\kappa) = p(log(\alpha) + log(\kappa)) +\sum_{i = 1}^{p} (\kappa-1)log(t_i) -\sum_{i = 1}^{n} \alpha t_{i}^{\kappa}\]

\textbf{Question 4:}

Calculons l'expression exacte du gradient de la log-vraisemblance:

La dérivée par rapport au paramètre \(\alpha\) vaut:
\[\dfrac{\partial l}{\partial \alpha} = \dfrac{p}{\alpha} - \sum_{i = 1}^{n} t_{i}^{\kappa}\]

La dérivée par rapport au paramètre \(\kappa\) vaut:
\[\dfrac{\partial l}{\partial \kappa} = \dfrac{p}{\kappa} +\sum_{i = 1}^{p}log(t_{i}) - \alpha\kappa\sum_{i = 1}^{n}t_{i}^{\kappa-1}\]

Ainsi, le gradient de la log-vraisemblance a pour expression:
\[\nabla l=\begin{pmatrix}
\dfrac{p}{\alpha} - \sum_{i = 1}^{n} t_{i}^{\kappa}\\
\dfrac{p}{\kappa} +\sum_{i = 1}^{p}log(t_{i}) - \alpha\kappa\sum_{i = 1}^{n}t_{i}^{\kappa-1}
\end{pmatrix}\]

La matrice hessienne a l'expression suivante :

\[\begin{pmatrix}
\dfrac{\partial ^{2}l}{\partial \alpha^{2}} & \dfrac{\partial^{2} l}{\partial\alpha\partial\kappa} \\ \dfrac{\partial^{2} l}{\partial\kappa\partial\alpha}
 & \dfrac{\partial ^{2}l}{\partial \kappa^{2}} \\
\end{pmatrix} = \begin{pmatrix} -\dfrac{p}{\alpha^{2}} & - \kappa\sum_{i = 1}^{n}t_{i}^{\kappa-1} \\ -\kappa \sum_{i = 1}^{n}t_{i}^{\kappa-1} & -\dfrac{p}{\kappa^{2}} -\alpha\sum_{i = 1}^{n}t_{i}^{\kappa-1} - \alpha \kappa (\kappa-1)\sum_{i = 1}^{n}t_{i}^{\kappa-2}   \\\end{pmatrix}\]

\textbf{\emph{Approche fréquentiste}}

\textbf{Question 5} Les estimateurs de maximum de vraisemblance
vérifient les relations suivantes
\[\dfrac{\partial l}{\partial \alpha} = 0\] et:
\[\dfrac{\partial l}{\partial \kappa} = 0\]

(De plus, le hessien est définie négatif pour les estimateurs du maximum
de vraisemblance.) On peut déduire de l'annulation de la dérivée par
rapport à \(\alpha\) la relation suivante:
\[\dfrac{p}{\hat{\alpha}_{MLE}} - \sum_{i = 1}^{n} t_{i}^{\hat{\kappa}_{MLE}} = 0\]

D'où l'on conclut que l'estimateur \(\alpha_{MLE}\) peut se déduire de
l'estimateur du maximum de vraisemblance \(\kappa_{MLE}\) à l'aide de la
formule suivante:

\[\hat{\alpha}_{MLE} = \dfrac{p}{\sum_{i = 1}^{n} t_{i}^{\hat{\kappa}_{MLE}}}\]
Par ailleurs, sachant que c'est cette valeur de \(\alpha\) qui maximise
la vraisemblance, on peut désormais maximiser la vraisemblance en
remplaçant \(\alpha\) par
\(\dfrac{p}{\sum_{i = 1}^{n} t_{i}^{\kappa}}\).

Aussi, pour trouver \(\hat{\kappa}_{MLE}\) il suffit de résoudre le
problème d'optimisation en 1D suivant:
\[\hat{\kappa}_{MLE}=argmax_{\kappa}(l(t_{1:n}| \dfrac{p}{\sum_{i = 1}^{n} t_{i}^{\kappa}},\kappa))\]

\textbf{Question 6:} On utilise la fonct \emph{optimize} pour trouver la
solution du problème précédent et on écrit la fonction \emph{maxvraise}
qui permet de calculer les estimateurs du maximum de vraisemblance de
(\(\alpha\),\(\kappa\)):

\begin{Shaded}
\begin{Highlighting}[]
\NormalTok{weibullcensLogLik<-}\ControlFlowTok{function}\NormalTok{(alpha,kappa,t,p)}
\NormalTok{\{}
  \KeywordTok{return}\NormalTok{ (p}\OperatorTok{*}\NormalTok{(}\KeywordTok{log}\NormalTok{(alpha)}\OperatorTok{+}\KeywordTok{log}\NormalTok{(kappa))}\OperatorTok{+}\NormalTok{(kappa}\DecValTok{-1}\NormalTok{)}\OperatorTok{*}\KeywordTok{sum}\NormalTok{(}\KeywordTok{log}\NormalTok{(t)[}\DecValTok{1}\OperatorTok{:}\NormalTok{p])}\OperatorTok{-}\NormalTok{alpha}\OperatorTok{*}\KeywordTok{sum}\NormalTok{(t}\OperatorTok{^}\NormalTok{kappa))}
\NormalTok{\}}
\NormalTok{maxvraise<-}\ControlFlowTok{function}\NormalTok{(t,p)\{}
\NormalTok{  kappaopt=}\StringTok{ }\KeywordTok{optimize}\NormalTok{(}\ControlFlowTok{function}\NormalTok{(kappa)\{}\KeywordTok{return}\NormalTok{ (}\KeywordTok{weibullcensLogLik}\NormalTok{(p}\OperatorTok{/}\KeywordTok{sum}\NormalTok{(t}\OperatorTok{^}\NormalTok{kappa),kappa,t,p))\},}\DataTypeTok{interval =} \KeywordTok{seq}\NormalTok{(}\DataTypeTok{from=}\DecValTok{0}\NormalTok{, }\DataTypeTok{to=}\DecValTok{20}\NormalTok{,}\DataTypeTok{length.out =} \DecValTok{100}\NormalTok{), }\DataTypeTok{maximum =} \OtherTok{TRUE}\NormalTok{)}\OperatorTok{$}\NormalTok{maximum}
\NormalTok{  alpha=p}\OperatorTok{/}\KeywordTok{sum}\NormalTok{(t}\OperatorTok{^}\NormalTok{kappaopt)}
  \KeywordTok{return}\NormalTok{ (}\KeywordTok{list}\NormalTok{(alpha,kappaopt))}
\NormalTok{\}}
\end{Highlighting}
\end{Shaded}

\textbf{Question 7:} On peut maintenant estimer (\(\alpha\),\(\kappa\))
par maiximum de vraisemblance 100 fois, à partir de 100 jeu de données
simulés selon la loi de Weibull censurée à \(t_0\):

\begin{Shaded}
\begin{Highlighting}[]
\NormalTok{alpha=}\DecValTok{5}
\NormalTok{kappa=}\DecValTok{2}
\NormalTok{t0=}\StringTok{ }\NormalTok{(}\OperatorTok{-}\KeywordTok{log}\NormalTok{(}\DecValTok{1}\FloatTok{-0.6}\NormalTok{)}\OperatorTok{/}\NormalTok{alpha)}\OperatorTok{^}\NormalTok{(}\DecValTok{1}\OperatorTok{/}\NormalTok{kappa)}
\NormalTok{n=}\DecValTok{1000}
\NormalTok{alphaest=}\KeywordTok{c}\NormalTok{()}
\NormalTok{kappaest=}\KeywordTok{c}\NormalTok{()}
\ControlFlowTok{for}\NormalTok{(i }\ControlFlowTok{in} \DecValTok{1}\OperatorTok{:}\DecValTok{100}\NormalTok{)}
\NormalTok{\{}
\NormalTok{  x=}\KeywordTok{simul_weibull_censur}\NormalTok{(n,alpha,kappa,t0)}
\NormalTok{  t=x[[}\DecValTok{1}\NormalTok{]]}
\NormalTok{  p=x[[}\DecValTok{2}\NormalTok{]]}
\NormalTok{  abmaxvr=}\KeywordTok{maxvraise}\NormalTok{(t,p)}
\NormalTok{  alphaest=}\KeywordTok{c}\NormalTok{(alphaest,abmaxvr[[}\DecValTok{1}\NormalTok{]])}
\NormalTok{  kappaest=}\KeywordTok{c}\NormalTok{(kappaest,abmaxvr[[}\DecValTok{2}\NormalTok{]])}
\NormalTok{\}}
\end{Highlighting}
\end{Shaded}

On calcule ensuite le biais de l'estimateur de alpha et de kappa:

\begin{Shaded}
\begin{Highlighting}[]
\NormalTok{biais_alpha=}\KeywordTok{mean}\NormalTok{(alphaest)}\OperatorTok{-}\NormalTok{alpha}
\NormalTok{biais_kappa=}\KeywordTok{mean}\NormalTok{(kappaest)}\OperatorTok{-}\NormalTok{kappa}
\NormalTok{biais_alpha}
\end{Highlighting}
\end{Shaded}

\begin{verbatim}
## [1] 0.01960185
\end{verbatim}

\begin{Shaded}
\begin{Highlighting}[]
\NormalTok{biais_kappa}
\end{Highlighting}
\end{Shaded}

\begin{verbatim}
## [1] 0.004087517
\end{verbatim}

Puis la variance des deux estimateurs:

\begin{Shaded}
\begin{Highlighting}[]
\NormalTok{var_alpha<-}\KeywordTok{mean}\NormalTok{((alphaest}\OperatorTok{-}\KeywordTok{mean}\NormalTok{(alphaest))}\OperatorTok{^}\DecValTok{2}\NormalTok{)}
\NormalTok{var_beta<-}\KeywordTok{mean}\NormalTok{((kappaest}\OperatorTok{-}\KeywordTok{mean}\NormalTok{(kappaest))}\OperatorTok{^}\DecValTok{2}\NormalTok{)}
\NormalTok{var_alpha}
\end{Highlighting}
\end{Shaded}

\begin{verbatim}
## [1] 0.1632623
\end{verbatim}

\begin{Shaded}
\begin{Highlighting}[]
\NormalTok{var_beta}
\end{Highlighting}
\end{Shaded}

\begin{verbatim}
## [1] 0.004837454
\end{verbatim}

Et enfin, on calcule leur coefficient de variation:

\begin{Shaded}
\begin{Highlighting}[]
\NormalTok{cv_alpha<-}\KeywordTok{mean}\NormalTok{((alphaest}\OperatorTok{-}\KeywordTok{mean}\NormalTok{(alphaest))}\OperatorTok{^}\DecValTok{2}\NormalTok{)}\OperatorTok{/}\KeywordTok{mean}\NormalTok{(alphaest)}
\NormalTok{cv_beta<-}\KeywordTok{mean}\NormalTok{((kappaest}\OperatorTok{-}\KeywordTok{mean}\NormalTok{(kappaest))}\OperatorTok{^}\DecValTok{2}\NormalTok{)}\OperatorTok{/}\KeywordTok{mean}\NormalTok{(kappaest)}
\NormalTok{cv_alpha}
\end{Highlighting}
\end{Shaded}

\begin{verbatim}
## [1] 0.03252496
\end{verbatim}

\begin{Shaded}
\begin{Highlighting}[]
\NormalTok{cv_beta}
\end{Highlighting}
\end{Shaded}

\begin{verbatim}
## [1] 0.002413794
\end{verbatim}

On retrouve bien un biais proche de zéro ce qui est logique pour un
estimateur du maximum de vraisemblance (asymptotiquement sans biais). De
plus, la variance semble tendre vers 0 ce qui est aussi cohérent avec
notre EMV. Le coefficient de variation est relativement faible ce qui
traduit une dispersion assez faible. La méthode de simulation semble
efficace

\textbf{\emph{Approche bayésienne}}

\textbf{Question 8:}

Montrer que si la loi a priori de \(\alpha\) est la distribution de
Gamma \(\mathcal G(a, b)\), alors la loi a posteriori conditionnelle de
\(\alpha\) sachant κ, notée \(\pi(α|κ, t_{1:n}, p)\) est encore une loi
Gamma, dont on précisera les paramètres.

Le théorème de Bayes nous donne la relation suivante :
\(\pi(\alpha|\kappa,t_{1:n},p) \propto \mathcal{L(t_{1:n}|\alpha,\kappa,p)}\times\pi(\alpha)\)

Supposons que : \[\pi(\alpha) \propto \alpha^{a-1}e^{-b\alpha}\]

Alors, la densité a posteriori de λ est proportionnelle vérifie
\(\pi(\alpha|\kappa,t_{1:n},p) \propto \mathcal{L(t_{1:n}|\alpha,\kappa,p)}\alpha^{a-1}e^{-b\alpha}\)

Or
\(\mathcal{L(t_{1:n}|\alpha,\kappa,p)} = \alpha^{p} \times \kappa^{p} \prod_{i = 1}^{p} t_{i}^{\kappa-1} \times e^{-\sum_{i = 1}^{n} \alpha t_{i}^{\kappa}} \propto a^{p}\times e^{-\sum_{i = 1}^{n} \alpha t_{i}^{\kappa}}\)

D'où
\(\pi(\alpha) \propto \alpha^{a-1}e^{-b\alpha} \times a^{p} e^{-\sum_{i = 1}^{n} \alpha t_{i}^{\kappa}}\)

D'où
\(\pi(\alpha) \propto \alpha^{a+p-1}e^{-\alpha(b+\sum_{i = 1}^{n} t_{i}^{\kappa})}\)

Ce qui montre que la loi a posteriori suit une loi Gamma
\(\mathcal{G(a+p,b+\sum_{i = 1}^{n} t_{i}^{\kappa})}\)

Dans la suite, on prendra de même une loi a priori de type Gamma pour
\(\kappa\), de paramètres c et d, et on utilisera le choix ``faiblement
informatif'' suivant : a = b = c = d = 10−3.

\textbf{Question 9}

Montrons que la densité marginale a posteriori de κ est proportionnelle
à :
\[\pi(\kappa|t_{1:n},p) \propto \pi(\kappa)\kappa^{p}\prod_{i=1}^{p}t_{i}^{\kappa-1}(b+\sum_{i = 1}^{n}t_{i}^{\kappa})^{-(a+p)}\]

On a la relation suivante :
\[\pi(\kappa|,t_{1:n},\alpha,p) \pi(\alpha |t_{1:n},p )  = \pi(\kappa | t_{1:n},p)\]

D'où
\[\pi(\kappa|,t_{1:n},p)  = \dfrac{\pi(\kappa,\alpha ,t_{1:n},p)}{\pi(\alpha | t_{1:n},p)}\]

On applique alors le théorème de Bayes :
\[\pi(\kappa,\alpha | t_{1:n},p) \propto \mathcal{L(t_{1:n}|\kappa,\alpha)}\pi(\kappa)\pi(\alpha)\]

D'où
\[\pi(\kappa | t_{1:n},p) \propto \dfrac{\pi(\alpha)\pi(\kappa) \mathcal{L(t_{1:n}|\kappa,\alpha)}}{\pi(\alpha|\kappa,t_{1:n})}\]

\[\pi(\kappa | t_{1:n},p) \propto \dfrac{\pi(\kappa)\alpha^{a-1}e^{-b\alpha}\kappa^{p}\prod_{i=1}^{p} t_{i}^{\kappa-1}e^{-\alpha\sum_{i=1}^{n}t_{i}^{\kappa}}}{\alpha^{a+p-1}e^{-\alpha(b+\sum_{i = 1}^{n} t_{i}^{\kappa})}(b+\sum_{i = 1}^{n} t_{i}^{\kappa})^{a+p}}\]
D'où
\[\pi(\kappa | t_{1:n},p) \propto \pi(\kappa) \kappa^{p} \prod_{i=1}^{p} t_{i}^{\kappa-1}(b+\sum_{i = 1}^{n} t_{i}^{\kappa})^{-(a+p)}\]

\textbf{(a)}

Montrons en supposant \(t_{max} = \max t_{i} > 1\) les équivalents
suivants (à une constante multiplicative près):

\[\pi(\kappa|t_{1:n},p) \sim_{\kappa\to0 } \pi(\kappa)\kappa^{p}\]
Lorsque \(\kappa \to 0\) on a
\[\prod_{i=1}^{p} t_{i}^{\kappa-1} \to \prod_{i=1}^{p} t_{i}^{-1} \] Et
\[(b+\sum_{i = 1}^{n} t_{i}^{\kappa})^{-(a+p)} \to (b+n)^{-(a+p)}\] Ces
termes ne dépendent plus de \(\kappa\), d'où le résultat souhaité :

\[\pi(\kappa|t_{1:n},p) \sim_{\kappa\to0 } \pi(\kappa)\kappa^{p} \]
Montrons désormais que:
\[\pi(\kappa|t_{1:n},p) \sim_{\kappa\to \infty } \pi(\kappa)t_{max}^{\kappa(a+n)}\]

Ce qui revient à montrer que pour \(\kappa\to \infty\) on a
\[\dfrac{\pi(\kappa|t_{1:n},p)}{\pi(\kappa)t_{max}^{\kappa(a+p)}} \to 1 \]

Pour \(\kappa\to \infty\) on a
\[\lim_{\kappa\to \infty} \dfrac{\pi(\kappa|t_{1:n},p)}{\pi(\kappa)\prod_{i=1}^{p}t_{i}^{\kappa-1}t_{max}^{\kappa(a+p)}} = \lim_{\kappa\to \infty} (\dfrac{b+\sum_{i = 1}^{n} t_{i}^{\kappa}}{t_{max}^{\kappa}})^{-(a+p)} = 1\]

Donc:
\[\pi(\kappa|t_{1:n},p) \sim  \pi(\kappa)\prod_{i=1}^{p}t_{i}^{\kappa-1}t_{max}^{\kappa(a+p)}\]

Déterminons les conditions sur e et f pour que le quotient avec
\(g(\kappa | e,f)\) reste borné. On veut que les limites au voisinage de
0 et \(+ \infty\) soient nulles pour avoir le rapport
\(\dfrac{\pi(\kappa| t_{1:n})}{g(\kappa)}\) soit borné sur
\(\mathbb{R}\)

\[\dfrac{\pi(\kappa| t_{1:n})}{g(\kappa)} \sim \dfrac{\kappa^{c-1}e^{-d\kappa}\kappa^{p}}{\kappa^{e-1}e^{-f\kappa}} \]
On doit avoir \(e \le p+c\) pour avoir
\(\dfrac{\pi(\kappa| t_{1:n})}{g(\kappa)}\) borné au \(\mathcal{V}(0)\)

On obtient la condition sur f en raisonnant de manière analogue pour
\(\kappa \to \infty\) On doit avoir le rapport
\(\dfrac{\pi(\kappa| t_{1:n})}{g(\kappa)}\) borné au
\(\mathcal{V}(\infty)\)

\[\dfrac{\pi(\kappa| t_{1:n})}{g(\kappa)} \sim \dfrac{\kappa^{c-1}e^{-d\kappa}\kappa^{p}e^{-\kappa (a+p) \log (t_{max})}e^{(\kappa-1)\sum_{i=1}^{p}\log(t_{i})}}{\kappa^{e-1}e^{-f\kappa}} \]

On doit avoir \(f < d + (a+p)\log(t_{max}) - \sum_{i=1}^{p}\log(t_{i})\)
ou \(f = d + (a+p)\log(t_{max}) - \sum_{i=1}^{p}\log(t_{i})\) et
\(e\geq c\) pour avoir \(\dfrac{\pi(\kappa| t_{1:n})}{g(\kappa)}\) borné
au \(\mathcal{V}(\infty)\)

\textbf{(b)} On effectue le calcul de \(\hat{\kappa}_{MAP}\) à l'aide de
la fonction \emph{optimize}:

\begin{Shaded}
\begin{Highlighting}[]
\NormalTok{a=}\FloatTok{1e-3}
\NormalTok{b=}\FloatTok{1e-3}
\NormalTok{c=}\FloatTok{1e-3}
\NormalTok{d=}\FloatTok{1e-3}
\NormalTok{p=}\DecValTok{6}
\NormalTok{n=}\DecValTok{10}
\NormalTok{t=}\KeywordTok{sort}\NormalTok{(df[,}\DecValTok{2}\NormalTok{])}
\NormalTok{neg_log_post =}\StringTok{ }\ControlFlowTok{function}\NormalTok{(kappa)\{}
   \OperatorTok{-}\StringTok{ }\KeywordTok{dgamma}\NormalTok{(kappa, c, d, }\DataTypeTok{log=}\OtherTok{TRUE}\NormalTok{) }\OperatorTok{+}\StringTok{ }\NormalTok{(a}\OperatorTok{+}\NormalTok{p) }\OperatorTok{*}\StringTok{ }\KeywordTok{log}\NormalTok{((b }\OperatorTok{+}\StringTok{ }\KeywordTok{sum}\NormalTok{(t}\OperatorTok{^}\NormalTok{kappa))) }\OperatorTok{-}\StringTok{ }\NormalTok{p }\OperatorTok{*}\StringTok{ }\KeywordTok{log}\NormalTok{(kappa) }\OperatorTok{-}\StringTok{ }\NormalTok{(kappa}\DecValTok{-1}\NormalTok{) }\OperatorTok{*}\StringTok{  }\KeywordTok{sum}\NormalTok{(}\KeywordTok{log}\NormalTok{(t[}\DecValTok{1}\OperatorTok{:}\NormalTok{p]))}
\NormalTok{\}}


\NormalTok{o =}\StringTok{ }\KeywordTok{optimize}\NormalTok{(neg_log_post, }\DataTypeTok{interval=}\KeywordTok{c}\NormalTok{(}\FloatTok{0.1}\NormalTok{, }\DecValTok{10}\NormalTok{) )}
\NormalTok{hat_kappa_MAP =}\StringTok{ }\NormalTok{o}\OperatorTok{$}\NormalTok{minimum }
\NormalTok{hat_kappa_MAP}
\end{Highlighting}
\end{Shaded}

\begin{verbatim}
## [1] 1.052583
\end{verbatim}

Par la suite, on calcule \(\hat{\sigma}_{\hat{\kappa}_{MAP}}^2\):

\begin{Shaded}
\begin{Highlighting}[]
\KeywordTok{library}\NormalTok{(numDeriv)}
\end{Highlighting}
\end{Shaded}

\begin{verbatim}
## Warning: package 'numDeriv' was built under R version 4.0.3
\end{verbatim}

\begin{Shaded}
\begin{Highlighting}[]
\NormalTok{hat_prec_kappa =}\StringTok{ }\KeywordTok{hessian}\NormalTok{(neg_log_post, hat_kappa_MAP)}
\NormalTok{hat_v_kappa =}\StringTok{ }\DecValTok{1}\OperatorTok{/}\NormalTok{hat_prec_kappa}
\NormalTok{hat_v_kappa}
\end{Highlighting}
\end{Shaded}

\begin{verbatim}
##          [,1]
## [1,] 0.169822
\end{verbatim}

\textbf{(c)}

On choisit la loi instrumentale \(g \sim \mathcal{G}(e,f)\) de manière à
ce qu'on ait les relations suivantes pour fixé l'espérance et la
variance de la loi comme demandé dans l'énoncé:

\[\dfrac{e}{f} = \hat{\kappa}_{MAP}\] et
\[\dfrac{e}{f^{2}} = \hat{\sigma}_{MAP}^2 \]

On obtient \[\dfrac{\hat{\kappa}_{MAP}}{\hat{\sigma}_{MAP}^2} = f \] et
\[\dfrac{\hat{\kappa}_{MAP}^{2}}{\hat{\sigma}_{MAP}^2}  = e\]

\begin{Shaded}
\begin{Highlighting}[]
\NormalTok{e =}\StringTok{ }\NormalTok{hat_kappa_MAP}\OperatorTok{^}\DecValTok{2} \OperatorTok{/}\StringTok{ }\NormalTok{hat_v_kappa}
\NormalTok{f =}\StringTok{ }\NormalTok{hat_kappa_MAP }\OperatorTok{/}\StringTok{ }\NormalTok{hat_v_kappa}
\CommentTok{# verification des conditions :}
\ControlFlowTok{if}\NormalTok{ (( e }\OperatorTok{<=}\StringTok{ }\NormalTok{p }\OperatorTok{+}\StringTok{ }\NormalTok{c ) }\OperatorTok{&}\StringTok{ }\NormalTok{((f }\OperatorTok{<}\StringTok{ }\NormalTok{(a}\OperatorTok{+}\NormalTok{p)}\OperatorTok{*}\KeywordTok{log}\NormalTok{(}\KeywordTok{max}\NormalTok{(t))}\OperatorTok{+}\NormalTok{d}\OperatorTok{-}\KeywordTok{sum}\NormalTok{(}\KeywordTok{log}\NormalTok{(t)[}\DecValTok{1}\OperatorTok{:}\NormalTok{p]))}\OperatorTok{|}\NormalTok{((e }\OperatorTok{>=}\StringTok{ }\NormalTok{c)}\OperatorTok{&}\NormalTok{(f }\OperatorTok{==}\StringTok{ }\NormalTok{(a}\OperatorTok{+}\NormalTok{p)}\OperatorTok{*}\KeywordTok{log}\NormalTok{(}\KeywordTok{max}\NormalTok{(t))}\OperatorTok{+}\NormalTok{d}\OperatorTok{-}\KeywordTok{sum}\NormalTok{(}\KeywordTok{log}\NormalTok{(t)[}\DecValTok{1}\OperatorTok{:}\NormalTok{p])))))\{}
 \KeywordTok{print}\NormalTok{(}\StringTok{"loi instrumentale conforme"}\NormalTok{)}
\NormalTok{\}}\ControlFlowTok{else}\NormalTok{\{}
 \KeywordTok{print}\NormalTok{(}\StringTok{"loi instrumentale non conforme"}\NormalTok{)}
\NormalTok{\}}
\end{Highlighting}
\end{Shaded}

\begin{verbatim}
## [1] "loi instrumentale non conforme"
\end{verbatim}

\textbf{(d)}

\begin{Shaded}
\begin{Highlighting}[]
\NormalTok{func =}\StringTok{ }\ControlFlowTok{function}\NormalTok{(kappa)\{}
\KeywordTok{return}\NormalTok{(}\OperatorTok{-}\KeywordTok{neg_log_post}\NormalTok{(kappa) }\OperatorTok{-}\StringTok{ }\KeywordTok{dgamma}\NormalTok{(kappa,e,f,}\DataTypeTok{log=}\OtherTok{TRUE}\NormalTok{))}
\CommentTok{#-neg_log_post(kappa)-dgamma(kappa,e,f,log=TRUE)}
\NormalTok{\} }
\end{Highlighting}
\end{Shaded}

\begin{Shaded}
\begin{Highlighting}[]
\NormalTok{o=}\KeywordTok{optimize}\NormalTok{(func,}\DataTypeTok{interval=}\KeywordTok{c}\NormalTok{(}\FloatTok{0.1}\NormalTok{,}\DecValTok{10}\NormalTok{), }\DataTypeTok{maximum=}\OtherTok{TRUE}\NormalTok{)}
\NormalTok{logM =}\StringTok{ }\NormalTok{o}\OperatorTok{$}\NormalTok{objective}
\NormalTok{M=}\KeywordTok{exp}\NormalTok{(logM)}
\NormalTok{M}
\end{Highlighting}
\end{Shaded}

\begin{verbatim}
## [1] 1.247696e-12
\end{verbatim}

\textbf{Question 10}

\begin{Shaded}
\begin{Highlighting}[]
\NormalTok{acceptation_rejet_kappa <-}\ControlFlowTok{function}\NormalTok{(n_simu)}
\NormalTok{\{}
\NormalTok{  n_accept=}\DecValTok{0}
\NormalTok{  n_generer=}\DecValTok{0}
\NormalTok{  X <-}\KeywordTok{c}\NormalTok{()}
  \ControlFlowTok{while}\NormalTok{ (n_accept}\OperatorTok{<}\NormalTok{n_simu)}
\NormalTok{  \{}
\NormalTok{    n_generer=n_generer}\OperatorTok{+}\DecValTok{1}
\NormalTok{    Xi <-}\KeywordTok{rgamma}\NormalTok{(}\DecValTok{1}\NormalTok{,e,f)}
\NormalTok{    ui <-}\KeywordTok{runif}\NormalTok{(}\DecValTok{1}\NormalTok{,}\DecValTok{0}\NormalTok{,}\DecValTok{1}\NormalTok{)}
\NormalTok{    pi =}\KeywordTok{exp}\NormalTok{(}\OperatorTok{-}\KeywordTok{neg_log_post}\NormalTok{(Xi)) }\CommentTok{#il faut calculer pi(k | ti,p)}

\NormalTok{    Ri <-}\StringTok{ }\NormalTok{pi}\OperatorTok{/}\NormalTok{(M}\OperatorTok{*}\KeywordTok{dgamma}\NormalTok{(Xi,e,f))}
    \ControlFlowTok{if}\NormalTok{(ui}\OperatorTok{<=}\NormalTok{Ri) }\CommentTok{#si ui est inferieur au rapport alors on conserve Xi}
\NormalTok{    \{}
\NormalTok{      n_accept=n_accept}\OperatorTok{+}\DecValTok{1}
\NormalTok{      X <-}\KeywordTok{c}\NormalTok{(X,Xi)}
\NormalTok{    \}}
\NormalTok{  \}}
  \KeywordTok{return}\NormalTok{ (}\KeywordTok{list}\NormalTok{(}\DataTypeTok{X=}\NormalTok{X, }\DataTypeTok{taux_acceptation=}\NormalTok{ n_simu}\OperatorTok{/}\NormalTok{n_generer))}
\NormalTok{\}}

\NormalTok{nsimu=}\DecValTok{10000}

\NormalTok{kappa_posteriori=}\KeywordTok{acceptation_rejet_kappa}\NormalTok{(nsimu)}
\NormalTok{alpha_posteriori=}\KeywordTok{c}\NormalTok{()}

\ControlFlowTok{for}\NormalTok{(kappa }\ControlFlowTok{in}\NormalTok{ kappa_posteriori}\OperatorTok{$}\NormalTok{X )}
\NormalTok{\{}
\NormalTok{  alpha=}\KeywordTok{rgamma}\NormalTok{(}\DecValTok{1}\NormalTok{,a}\OperatorTok{+}\NormalTok{p,b}\OperatorTok{+}\KeywordTok{sum}\NormalTok{(t}\OperatorTok{^}\NormalTok{kappa))}
\NormalTok{  alpha_posteriori=}\KeywordTok{c}\NormalTok{(alpha_posteriori,alpha)}
\NormalTok{\}}
\end{Highlighting}
\end{Shaded}

\begin{Shaded}
\begin{Highlighting}[]
\KeywordTok{par}\NormalTok{(}\DataTypeTok{mfrow=}\KeywordTok{c}\NormalTok{(}\DecValTok{1}\NormalTok{,}\DecValTok{2}\NormalTok{))}
\KeywordTok{hist}\NormalTok{(kappa_posteriori}\OperatorTok{$}\NormalTok{X, }\DataTypeTok{breaks=}\StringTok{"Scott"}\NormalTok{, }\DataTypeTok{probability=}\OtherTok{TRUE}\NormalTok{, }\DataTypeTok{main=}\StringTok{""}\NormalTok{, }\DataTypeTok{xlab=}\KeywordTok{TeX}\NormalTok{(}\StringTok{'$}\CharTok{\textbackslash{}\textbackslash{}}\StringTok{kappa$'}\NormalTok{))}
\KeywordTok{curve}\NormalTok{(}\KeywordTok{dgamma}\NormalTok{(x,c,d), }\DataTypeTok{col=}\StringTok{'red'}\NormalTok{, }\DataTypeTok{add=}\OtherTok{TRUE}\NormalTok{)}

\KeywordTok{hist}\NormalTok{(alpha_posteriori, }\DataTypeTok{breaks=}\StringTok{"Scott"}\NormalTok{, }\DataTypeTok{probability=}\OtherTok{TRUE}\NormalTok{, }\DataTypeTok{main=}\StringTok{""}\NormalTok{, }\DataTypeTok{xlab=}\KeywordTok{TeX}\NormalTok{(}\StringTok{'$}\CharTok{\textbackslash{}\textbackslash{}}\StringTok{alpha$'}\NormalTok{))}
\KeywordTok{curve}\NormalTok{(}\KeywordTok{dgamma}\NormalTok{(x,a,b), }\DataTypeTok{col=}\StringTok{'red'}\NormalTok{, }\DataTypeTok{add=}\OtherTok{TRUE}\NormalTok{)}
\end{Highlighting}
\end{Shaded}

\includegraphics{sta211_files/figure-latex/unnamed-chunk-14-1.pdf}

\begin{Shaded}
\begin{Highlighting}[]
\KeywordTok{plot}\NormalTok{(kappa_posteriori}\OperatorTok{$}\NormalTok{X,alpha_posteriori}\OperatorTok{^}\NormalTok{\{}\OperatorTok{-}\DecValTok{1}\OperatorTok{/}\NormalTok{kappa_posteriori}\OperatorTok{$}\NormalTok{X\}, }\DataTypeTok{col=}\StringTok{'blue'}\NormalTok{, }\DataTypeTok{main=}\StringTok{"Graphe }
\StringTok{des correlations a posteriori"}\NormalTok{, }\DataTypeTok{xlab=}\KeywordTok{TeX}\NormalTok{(}\StringTok{"$}\CharTok{\textbackslash{}\textbackslash{}}\StringTok{kappa$"}\NormalTok{), }\DataTypeTok{ylab=}\KeywordTok{TeX}\NormalTok{(}\StringTok{"$}\CharTok{\textbackslash{}\textbackslash{}}\StringTok{alpha^\{-1/}\CharTok{\textbackslash{}\textbackslash{}}\StringTok{kappa\}$"}\NormalTok{))}
\end{Highlighting}
\end{Shaded}

\includegraphics{sta211_files/figure-latex/unnamed-chunk-15-1.pdf}

\textbf{Question 11:}

\begin{Shaded}
\begin{Highlighting}[]
\KeywordTok{library}\NormalTok{(}\StringTok{"coda"}\NormalTok{)}
\end{Highlighting}
\end{Shaded}

\begin{verbatim}
## Warning: package 'coda' was built under R version 4.0.5
\end{verbatim}

\begin{Shaded}
\begin{Highlighting}[]
\NormalTok{dweibull<-}\ControlFlowTok{function}\NormalTok{(alpha,kappa,t,p)}
\NormalTok{\{}
  \KeywordTok{return}\NormalTok{ (((alpha}\OperatorTok{*}\NormalTok{kappa)}\OperatorTok{^}\NormalTok{p)}\OperatorTok{*}\NormalTok{((}\KeywordTok{prod}\NormalTok{(t[}\DecValTok{1}\OperatorTok{:}\NormalTok{p]))}\OperatorTok{^}\NormalTok{(kappa}\DecValTok{-1}\NormalTok{))}\OperatorTok{*}\NormalTok{(}\KeywordTok{exp}\NormalTok{(}\OperatorTok{-}\NormalTok{alpha}\OperatorTok{*}\KeywordTok{sum}\NormalTok{(t}\OperatorTok{^}\NormalTok{kappa))))}
\NormalTok{\}}
\NormalTok{pi <-}\StringTok{ }\ControlFlowTok{function}\NormalTok{(x)}
\NormalTok{\{}
  \KeywordTok{return}\NormalTok{(}\KeywordTok{dgamma}\NormalTok{(x,c,d)}\OperatorTok{*}\NormalTok{(x}\OperatorTok{^}\NormalTok{p)}\OperatorTok{*}\KeywordTok{exp}\NormalTok{((x}\DecValTok{-1}\NormalTok{)}\OperatorTok{*}\KeywordTok{sum}\NormalTok{(}\KeywordTok{log}\NormalTok{(t)))}\OperatorTok{*}\NormalTok{(b}\OperatorTok{+}\KeywordTok{sum}\NormalTok{(t}\OperatorTok{^}\NormalTok{x))}\OperatorTok{^}\NormalTok{(}\OperatorTok{-}\NormalTok{(a}\OperatorTok{+}\NormalTok{p)))}
\NormalTok{\}}

\NormalTok{MCMC<-}\ControlFlowTok{function}\NormalTok{(t,p,iter,sdnorm,alpha0,kappa0)\{}
  \CommentTok{##Taux d'acceptation}
\NormalTok{  accept=}\KeywordTok{vector}\NormalTok{(}\StringTok{"numeric"}\NormalTok{,iter)}
  \CommentTok{##Valeur courante}
\NormalTok{  alphac=alpha0}
\NormalTok{  kappac=kappa0}
  \CommentTok{##Création de la chaîne de Markov}
\NormalTok{  alpha_chain=alphac}
\NormalTok{  kappachain=kappac}
  \ControlFlowTok{for}\NormalTok{(i }\ControlFlowTok{in} \DecValTok{1}\OperatorTok{:}\NormalTok{(iter}\DecValTok{-1}\NormalTok{))}
\NormalTok{  \{}
    \CommentTok{##Mise à jour}
    \CommentTok{##Gibbs}
\NormalTok{    alphacandidat=}\KeywordTok{rgamma}\NormalTok{(}\DecValTok{1}\NormalTok{,a}\OperatorTok{+}\NormalTok{p,b}\OperatorTok{+}\KeywordTok{sum}\NormalTok{(t}\OperatorTok{^}\NormalTok{kappac))}
    \CommentTok{#Metropolis-Hastings}
\NormalTok{    kappacandidat=}\KeywordTok{rnorm}\NormalTok{(}\DecValTok{1}\NormalTok{,}\DataTypeTok{mean=}\NormalTok{kappac,}\DataTypeTok{sd=}\NormalTok{sdnorm)}
\NormalTok{    r=(}\KeywordTok{dgamma}\NormalTok{(kappacandidat,c,d)}\OperatorTok{/}\KeywordTok{dgamma}\NormalTok{(kappac,c,d))}\OperatorTok{*}\NormalTok{(}\KeywordTok{dweibull}\NormalTok{(alphacandidat,kappacandidat,t,p)}\OperatorTok{/}\KeywordTok{dweibull}\NormalTok{(alphac,kappac,t,p))}\OperatorTok{*}\NormalTok{(}\KeywordTok{dnorm}\NormalTok{(kappac,}\DataTypeTok{mean=}\NormalTok{kappacandidat,}\DataTypeTok{sd=}\NormalTok{sdnorm)}\OperatorTok{/}\KeywordTok{dnorm}\NormalTok{(kappacandidat,}\DataTypeTok{mean=}\NormalTok{kappac,}\DataTypeTok{sd=}\NormalTok{sdnorm))}
    \CommentTok{#r=(pi(kappacandidat)/pi(kappac))*(dnorm(kappac,mean=kappacandidat,sd=sdnorm)/dnorm(kappacandidat,mean=kappac,sd=sdnorm))}
\NormalTok{    u=}\KeywordTok{runif}\NormalTok{(}\DecValTok{1}\NormalTok{,}\DecValTok{0}\NormalTok{,}\DecValTok{1}\NormalTok{)}
\NormalTok{    minr1=}\KeywordTok{min}\NormalTok{(r,}\DecValTok{1}\NormalTok{)}
    \ControlFlowTok{if}\NormalTok{((}\OperatorTok{!}\KeywordTok{is.nan}\NormalTok{(minr1))}\OperatorTok{&}\NormalTok{(u}\OperatorTok{<}\NormalTok{minr1))}
\NormalTok{    \{}
\NormalTok{      accept[i]=}\DecValTok{1}
\NormalTok{      kappac=kappacandidat}
\NormalTok{    \}}
\NormalTok{    alphac=alphacandidat}
    \CommentTok{##Stockage des données}
\NormalTok{    alpha_chain=}\KeywordTok{c}\NormalTok{(alpha_chain,alphac)}
\NormalTok{    kappachain=}\KeywordTok{c}\NormalTok{(kappachain,kappac)}
\NormalTok{    \}}
  \KeywordTok{return}\NormalTok{ (}\KeywordTok{list}\NormalTok{(}\DataTypeTok{acceptrate=}\KeywordTok{mean}\NormalTok{(accept),}\DataTypeTok{result=}\KeywordTok{mcmc}\NormalTok{(}\KeywordTok{matrix}\NormalTok{(}\KeywordTok{c}\NormalTok{(kappachain,alpha_chain),}\DataTypeTok{nrow=}\NormalTok{iter,}\DataTypeTok{ncol=}\DecValTok{2}\NormalTok{,}\DataTypeTok{byrow =} \OtherTok{FALSE}\NormalTok{))))}
\NormalTok{\}}


\NormalTok{G=}\DecValTok{10000}
\NormalTok{delta=}\KeywordTok{seq}\NormalTok{(}\FloatTok{0.1}\NormalTok{,}\DecValTok{10}\NormalTok{,}\DataTypeTok{by=}\FloatTok{0.1}\NormalTok{)}
\NormalTok{acceptrates=}\KeywordTok{lapply}\NormalTok{(delta,}\ControlFlowTok{function}\NormalTok{(sdnorm)\{}\KeywordTok{return}\NormalTok{ (}\KeywordTok{MCMC}\NormalTok{(t,p,G, sdnorm,}\DecValTok{1}\NormalTok{,}\DecValTok{1}\NormalTok{)}\OperatorTok{$}\NormalTok{acceptrate)\})}

\KeywordTok{plot}\NormalTok{(delta,}\KeywordTok{unlist}\NormalTok{(acceptrates),}\DataTypeTok{ylab=}\StringTok{"Taux d'acceptation"}\NormalTok{)}
\KeywordTok{abline}\NormalTok{(}\DataTypeTok{h=}\FloatTok{0.4}\NormalTok{,}\DataTypeTok{col=}\StringTok{'red'}\NormalTok{)}
\end{Highlighting}
\end{Shaded}

\includegraphics{sta211_files/figure-latex/unnamed-chunk-16-1.pdf}

On obtient la valeur de delta suivante permettant d'obtenir un taux
d'acceptation de 40\%:

\begin{Shaded}
\begin{Highlighting}[]
\NormalTok{acceptrates[}\KeywordTok{which.min}\NormalTok{(}\KeywordTok{abs}\NormalTok{(}\FloatTok{0.4}\OperatorTok{-}\KeywordTok{as.numeric}\NormalTok{(acceptrates)))]}
\end{Highlighting}
\end{Shaded}

\begin{verbatim}
## [[1]]
## [1] 0.4251
\end{verbatim}

\begin{Shaded}
\begin{Highlighting}[]
\NormalTok{deltaopt=delta[}\KeywordTok{which.min}\NormalTok{(}\KeywordTok{abs}\NormalTok{(}\FloatTok{0.4}\OperatorTok{-}\KeywordTok{as.numeric}\NormalTok{(acceptrates)))]}
\end{Highlighting}
\end{Shaded}

\textbf{Question 13}

\begin{Shaded}
\begin{Highlighting}[]
\NormalTok{chain1 <-}\StringTok{ }\KeywordTok{MCMC}\NormalTok{(t,p,}\DecValTok{10000}\NormalTok{, deltaopt,}\FloatTok{0.5}\NormalTok{,}\FloatTok{0.5}\NormalTok{)}\OperatorTok{$}\NormalTok{result}
\NormalTok{chain2 <-}\StringTok{ }\KeywordTok{MCMC}\NormalTok{(t,p,}\DecValTok{10000}\NormalTok{, deltaopt,}\FloatTok{1.5}\NormalTok{,}\DecValTok{1}\NormalTok{)}\OperatorTok{$}\NormalTok{result}
\NormalTok{chain3 <-}\StringTok{ }\KeywordTok{MCMC}\NormalTok{(t,p,}\DecValTok{10000}\NormalTok{, deltaopt,}\DecValTok{1}\NormalTok{,}\DecValTok{1}\NormalTok{)}\OperatorTok{$}\NormalTok{result}
\NormalTok{mcmcchains <-}\StringTok{ }\KeywordTok{mcmc.list}\NormalTok{(}\KeywordTok{list}\NormalTok{(chain1,chain2,chain3))}
\KeywordTok{plot}\NormalTok{(mcmcchains)}
\end{Highlighting}
\end{Shaded}

\includegraphics{sta211_files/figure-latex/unnamed-chunk-18-1.pdf}

\begin{Shaded}
\begin{Highlighting}[]
\KeywordTok{gelman.diag}\NormalTok{(mcmcchains)}
\end{Highlighting}
\end{Shaded}

\begin{verbatim}
## Potential scale reduction factors:
## 
##      Point est. Upper C.I.
## [1,]          1          1
## [2,]          1          1
## 
## Multivariate psrf
## 
## 1
\end{verbatim}

\begin{Shaded}
\begin{Highlighting}[]
\KeywordTok{gelman.plot}\NormalTok{(mcmcchains)}
\KeywordTok{abline}\NormalTok{(}\DataTypeTok{h=}\FloatTok{1.05}\NormalTok{, }\DataTypeTok{col=}\StringTok{"orange"}\NormalTok{)}
\end{Highlighting}
\end{Shaded}

\includegraphics{sta211_files/figure-latex/unnamed-chunk-18-2.pdf}

Tout d'abord, on peut remarquer que les chaînes se mélangent
correctement, ainsi il n'y a pas d'indice de non convergence de
l'algorithme.

De plus, on ne peut pas vraiment conclure sur la convergence avec le
critère de Gelman-Rubin, en effet il semble qu'il y ait stabilité mais
il y a quelques pics pour les premières itérations. Il y a peut être un
très léger problème de convergence au départ car en appliquant le
critère de Gelman-Rubin, on doit obtenir stabilité et avoir
R\textless1.05. De plus, il faut noter que la vitesse de convergence
dépend du point de départ. Une solution est alors d'augmenter le nombre
d'itérations comme nous le faisons ci-après avec 100 000 itérations.

De plus, pour atteindre l'état stationnaire, on peut supposer d'après
les courbes ci-dessus que l'on peut prendre à minima 2000 itérations.

Par ailleurs, ci-dessous on augmente le nombre d'itérations pour
vérifier l'amélioration de la convergence de l'algorithme:

\begin{Shaded}
\begin{Highlighting}[]
\NormalTok{chain1 <-}\StringTok{ }\KeywordTok{MCMC}\NormalTok{(t,p,}\DecValTok{100000}\NormalTok{, deltaopt,}\FloatTok{0.5}\NormalTok{,}\FloatTok{0.5}\NormalTok{)}\OperatorTok{$}\NormalTok{result}
\NormalTok{chain2 <-}\StringTok{ }\KeywordTok{MCMC}\NormalTok{(t,p,}\DecValTok{100000}\NormalTok{, deltaopt,}\FloatTok{1.5}\NormalTok{,}\DecValTok{1}\NormalTok{)}\OperatorTok{$}\NormalTok{result}
\NormalTok{chain3 <-}\StringTok{ }\KeywordTok{MCMC}\NormalTok{(t,p,}\DecValTok{100000}\NormalTok{, deltaopt,}\DecValTok{1}\NormalTok{,}\DecValTok{1}\NormalTok{)}\OperatorTok{$}\NormalTok{result}
\NormalTok{mcmcchains <-}\StringTok{ }\KeywordTok{mcmc.list}\NormalTok{(}\KeywordTok{list}\NormalTok{(chain1,chain2,chain3))}
\KeywordTok{plot}\NormalTok{(mcmcchains, }\DataTypeTok{density=}\OtherTok{TRUE}\NormalTok{)}
\end{Highlighting}
\end{Shaded}

\includegraphics{sta211_files/figure-latex/unnamed-chunk-19-1.pdf}

\begin{Shaded}
\begin{Highlighting}[]
\KeywordTok{gelman.diag}\NormalTok{(mcmcchains)}
\end{Highlighting}
\end{Shaded}

\begin{verbatim}
## Potential scale reduction factors:
## 
##      Point est. Upper C.I.
## [1,]          1          1
## [2,]          1          1
## 
## Multivariate psrf
## 
## 1
\end{verbatim}

\begin{Shaded}
\begin{Highlighting}[]
\KeywordTok{gelman.plot}\NormalTok{(mcmcchains)}
\end{Highlighting}
\end{Shaded}

\includegraphics{sta211_files/figure-latex/unnamed-chunk-19-2.pdf}

Tout d'abord, on peut remarquer que les chaînes se mélangent
correctement, ainsi il n'y a pas d'indice de non convergence de
l'algorithme.

De plus, en utilisant le critère de Gelman-Rubin, on peut appliquer la
règle standard car le graphe nous montre qu'il y a stabilité et on peut
remarquer qu'on a R\textless1.05 à partir d'un certain rang, donc il n'y
a pas de problème de convergence majeure désormais.

\textbf{Question 14}

On resimule les 10000 réalisations dans un premier temps:

\begin{Shaded}
\begin{Highlighting}[]
\NormalTok{G1 <-}\StringTok{ }\DecValTok{10000}

\NormalTok{chain1_G1 <-}\StringTok{ }\KeywordTok{MCMC}\NormalTok{(t,p,}\DecValTok{10000}\NormalTok{, deltaopt,}\FloatTok{0.5}\NormalTok{,}\FloatTok{0.5}\NormalTok{)}\OperatorTok{$}\NormalTok{result}
\NormalTok{chain2_G1 <-}\StringTok{ }\KeywordTok{MCMC}\NormalTok{(t,p,}\DecValTok{10000}\NormalTok{, deltaopt,}\FloatTok{1.5}\NormalTok{,}\DecValTok{1}\NormalTok{)}\OperatorTok{$}\NormalTok{result}
\NormalTok{chain3_G1 <-}\StringTok{ }\KeywordTok{MCMC}\NormalTok{(t,p,}\DecValTok{10000}\NormalTok{, deltaopt,}\DecValTok{1}\NormalTok{,}\DecValTok{1}\NormalTok{)}\OperatorTok{$}\NormalTok{result}
\NormalTok{mcmcchains <-}\StringTok{ }\KeywordTok{mcmc.list}\NormalTok{(}\KeywordTok{list}\NormalTok{(chain1_G1,chain2_G1,chain3_G1))}
\KeywordTok{plot}\NormalTok{(mcmcchains)}
\end{Highlighting}
\end{Shaded}

\includegraphics{sta211_files/figure-latex/unnamed-chunk-20-1.pdf}

On supprime désormais les itérations correspondant au temps de chauffe,
et par ailleurs on effectue les différents calculs avec un temps de
chauffe à 4000:

\begin{Shaded}
\begin{Highlighting}[]
\NormalTok{mcmctc_G1=}\KeywordTok{mcmc.list}\NormalTok{(}\KeywordTok{list}\NormalTok{(}\KeywordTok{as.mcmc}\NormalTok{(chain1_G1[}\DecValTok{2001}\OperatorTok{:}\DecValTok{10000}\NormalTok{,]),}\KeywordTok{as.mcmc}\NormalTok{(chain2_G1[}\DecValTok{2001}\OperatorTok{:}\DecValTok{10000}\NormalTok{,]),}\KeywordTok{as.mcmc}\NormalTok{(chain3_G1[}\DecValTok{2001}\OperatorTok{:}\DecValTok{10000}\NormalTok{,])))}
\NormalTok{ESS_G1=}\KeywordTok{effectiveSize}\NormalTok{(mcmctc_G1)}
\KeywordTok{names}\NormalTok{(ESS_G1)=}\KeywordTok{list}\NormalTok{(}\StringTok{"kappa"}\NormalTok{,}\StringTok{"alpha"}\NormalTok{)}
\NormalTok{ESS_G1}
\end{Highlighting}
\end{Shaded}

\begin{verbatim}
##    kappa    alpha 
## 1702.207 3073.148
\end{verbatim}

La taille de l'ESS est faible, c'est peut-être dû à une grande
auto-corrélation pour nos différentes chaînes MCMC qui réduit fortement
la taille de l'échantillon a posteriori.On augmente donc par la suite le
nombre d'itérations G pour pouvoir calculer les densités a posteriori
avec plus de précision, on effectue donc 100000 itérations, de plus on
choisit cette fois un temps de chauffe à 25000 afin de ne conserver que
les observations les plus précises:

\begin{Shaded}
\begin{Highlighting}[]
\NormalTok{mcmctc=}\KeywordTok{mcmc.list}\NormalTok{(}\KeywordTok{list}\NormalTok{(}\KeywordTok{as.mcmc}\NormalTok{(chain1[}\DecValTok{25001}\OperatorTok{:}\DecValTok{100000}\NormalTok{,]),}\KeywordTok{as.mcmc}\NormalTok{(chain2[}\DecValTok{25001}\OperatorTok{:}\DecValTok{100000}\NormalTok{,]),}\KeywordTok{as.mcmc}\NormalTok{(chain3[}\DecValTok{25001}\OperatorTok{:}\DecValTok{100000}\NormalTok{,])))}

\NormalTok{ESS=}\KeywordTok{effectiveSize}\NormalTok{(mcmctc)}

\KeywordTok{names}\NormalTok{(ESS)=}\KeywordTok{list}\NormalTok{(}\StringTok{"kappa"}\NormalTok{,}\StringTok{"alpha"}\NormalTok{)}

\NormalTok{ESS}
\end{Highlighting}
\end{Shaded}

\begin{verbatim}
##    kappa    alpha 
## 15777.26 27134.35
\end{verbatim}

On peut remarquer que l' ESS est désormais supérieur à celui que l'on
avait obtenu précédemment et on peut donc considérer que l'ESS est
maintenant ``satisfaisant''.

\textbf{Question 15}

On représente ci-dessous les densités a priori et a posteriori pour
chaque paramètre, ainsi que le graphe de corrélation a posteriori du
couple \((\kappa,\alpha^{-1/\kappa})\) puis on compare les résultats
issus des deux algorithmes d'inférence bayésienne:

Voici le résultats en termes de densités a priori et a posteriori pour
l'algorithme MCMC:

\begin{Shaded}
\begin{Highlighting}[]
\KeywordTok{par}\NormalTok{(}\DataTypeTok{mfrow=}\KeywordTok{c}\NormalTok{(}\DecValTok{1}\NormalTok{,}\DecValTok{2}\NormalTok{))}
\KeywordTok{plot}\NormalTok{(}\KeywordTok{mcmc.list}\NormalTok{(}\KeywordTok{list}\NormalTok{(}\KeywordTok{as.mcmc}\NormalTok{(chain1[}\DecValTok{25001}\OperatorTok{:}\DecValTok{100000}\NormalTok{,}\DecValTok{1}\NormalTok{]),}\KeywordTok{as.mcmc}\NormalTok{(chain2[}\DecValTok{25001}\OperatorTok{:}\DecValTok{100000}\NormalTok{,}\DecValTok{1}\NormalTok{]),}\KeywordTok{as.mcmc}\NormalTok{(chain3[}\DecValTok{25001}\OperatorTok{:}\DecValTok{100000}\NormalTok{,}\DecValTok{1}\NormalTok{]))),}\DataTypeTok{trace=}\OtherTok{FALSE}\NormalTok{,}\DataTypeTok{xlab=}\KeywordTok{TeX}\NormalTok{(}\StringTok{'$}\CharTok{\textbackslash{}\textbackslash{}}\StringTok{kappa$'}\NormalTok{),}\DataTypeTok{ylab=}\StringTok{"Density"}\NormalTok{)}
\KeywordTok{curve}\NormalTok{(}\KeywordTok{dgamma}\NormalTok{(x,c,d), }\DataTypeTok{col=}\StringTok{'red'}\NormalTok{, }\DataTypeTok{add=}\OtherTok{TRUE}\NormalTok{)}
\end{Highlighting}
\end{Shaded}

\includegraphics{sta211_files/figure-latex/unnamed-chunk-23-1.pdf}

\begin{Shaded}
\begin{Highlighting}[]
\KeywordTok{plot}\NormalTok{(}\KeywordTok{mcmc.list}\NormalTok{(}\KeywordTok{list}\NormalTok{(}\KeywordTok{as.mcmc}\NormalTok{(chain1[}\DecValTok{25001}\OperatorTok{:}\DecValTok{100000}\NormalTok{,}\DecValTok{2}\NormalTok{]),}\KeywordTok{as.mcmc}\NormalTok{(chain2[}\DecValTok{25001}\OperatorTok{:}\DecValTok{100000}\NormalTok{,}\DecValTok{2}\NormalTok{]),}\KeywordTok{as.mcmc}\NormalTok{(chain3[}\DecValTok{25001}\OperatorTok{:}\DecValTok{100000}\NormalTok{,}\DecValTok{2}\NormalTok{]))),}\DataTypeTok{trace=}\OtherTok{FALSE}\NormalTok{, }\DataTypeTok{xlab=}\KeywordTok{TeX}\NormalTok{(}\StringTok{'$}\CharTok{\textbackslash{}\textbackslash{}}\StringTok{alpha$'}\NormalTok{),}\DataTypeTok{ylab=}\StringTok{"Density"}\NormalTok{)}
\KeywordTok{curve}\NormalTok{(}\KeywordTok{dgamma}\NormalTok{(x,a,b), }\DataTypeTok{col=}\StringTok{'red'}\NormalTok{, }\DataTypeTok{add=}\OtherTok{TRUE}\NormalTok{)}
\end{Highlighting}
\end{Shaded}

\includegraphics{sta211_files/figure-latex/unnamed-chunk-23-2.pdf}

En traçant individuellement les résultats pour les trois chaines, on
peut aussi voir qu'ils concordent c'est un bon signe de convergence.

Comparaisons des deux algorithmes d'inférences bayesiens:

On a les résultats suivants:

\begin{Shaded}
\begin{Highlighting}[]
\KeywordTok{par}\NormalTok{(}\DataTypeTok{mfrow=}\KeywordTok{c}\NormalTok{(}\DecValTok{1}\NormalTok{,}\DecValTok{2}\NormalTok{))}
\KeywordTok{hist}\NormalTok{(kappa_posteriori}\OperatorTok{$}\NormalTok{X, }\DataTypeTok{breaks=}\StringTok{"Scott"}\NormalTok{, }\DataTypeTok{probability=}\OtherTok{TRUE}\NormalTok{, }\DataTypeTok{main=}\StringTok{""}\NormalTok{, }\DataTypeTok{xlab=}\KeywordTok{TeX}\NormalTok{(}\StringTok{'$}\CharTok{\textbackslash{}\textbackslash{}}\StringTok{kappa$'}\NormalTok{))}
\KeywordTok{curve}\NormalTok{(}\KeywordTok{dgamma}\NormalTok{(x,c,d), }\DataTypeTok{col=}\StringTok{'red'}\NormalTok{, }\DataTypeTok{add=}\OtherTok{TRUE}\NormalTok{)}

\KeywordTok{hist}\NormalTok{(alpha_posteriori, }\DataTypeTok{breaks=}\StringTok{"Scott"}\NormalTok{, }\DataTypeTok{probability=}\OtherTok{TRUE}\NormalTok{, }\DataTypeTok{main=}\StringTok{""}\NormalTok{, }\DataTypeTok{xlab=}\KeywordTok{TeX}\NormalTok{(}\StringTok{'$}\CharTok{\textbackslash{}\textbackslash{}}\StringTok{alpha$'}\NormalTok{))}
\KeywordTok{curve}\NormalTok{(}\KeywordTok{dgamma}\NormalTok{(x,a,b), }\DataTypeTok{col=}\StringTok{'red'}\NormalTok{, }\DataTypeTok{add=}\OtherTok{TRUE}\NormalTok{)}
\end{Highlighting}
\end{Shaded}

\includegraphics{sta211_files/figure-latex/unnamed-chunk-24-1.pdf}

\begin{Shaded}
\begin{Highlighting}[]
\KeywordTok{plot}\NormalTok{(kappa_posteriori}\OperatorTok{$}\NormalTok{X,alpha_posteriori}\OperatorTok{^}\NormalTok{\{}\OperatorTok{-}\DecValTok{1}\OperatorTok{/}\NormalTok{kappa_posteriori}\OperatorTok{$}\NormalTok{X\}, }\DataTypeTok{col=}\StringTok{'blue'}\NormalTok{, }\DataTypeTok{main=}\StringTok{"Graphe }
\StringTok{des correlations a posteriori"}\NormalTok{, }\DataTypeTok{xlab=}\KeywordTok{TeX}\NormalTok{(}\StringTok{"$}\CharTok{\textbackslash{}\textbackslash{}}\StringTok{kappa$"}\NormalTok{), }\DataTypeTok{ylab=}\KeywordTok{TeX}\NormalTok{(}\StringTok{"$}\CharTok{\textbackslash{}\textbackslash{}}\StringTok{alpha^\{-1/}\CharTok{\textbackslash{}\textbackslash{}}\StringTok{kappa\}$"}\NormalTok{))}


\KeywordTok{par}\NormalTok{(}\DataTypeTok{mfrow=}\KeywordTok{c}\NormalTok{(}\DecValTok{1}\NormalTok{,}\DecValTok{2}\NormalTok{))}
\end{Highlighting}
\end{Shaded}

\includegraphics{sta211_files/figure-latex/unnamed-chunk-24-2.pdf}

\begin{Shaded}
\begin{Highlighting}[]
\KeywordTok{plot}\NormalTok{(}\KeywordTok{mcmc.list}\NormalTok{(}\KeywordTok{list}\NormalTok{(}\KeywordTok{as.mcmc}\NormalTok{(chain1[}\DecValTok{25001}\OperatorTok{:}\DecValTok{100000}\NormalTok{,}\DecValTok{1}\NormalTok{]),}\KeywordTok{as.mcmc}\NormalTok{(chain2[}\DecValTok{25001}\OperatorTok{:}\DecValTok{100000}\NormalTok{,}\DecValTok{1}\NormalTok{]),}\KeywordTok{as.mcmc}\NormalTok{(chain3[}\DecValTok{25001}\OperatorTok{:}\DecValTok{100000}\NormalTok{,}\DecValTok{1}\NormalTok{]))),}\DataTypeTok{trace=}\OtherTok{FALSE}\NormalTok{,}\DataTypeTok{xlab=}\KeywordTok{TeX}\NormalTok{(}\StringTok{'$}\CharTok{\textbackslash{}\textbackslash{}}\StringTok{kappa$'}\NormalTok{),}\DataTypeTok{ylab=}\StringTok{"Density"}\NormalTok{)}
\KeywordTok{curve}\NormalTok{(}\KeywordTok{dgamma}\NormalTok{(x,c,d), }\DataTypeTok{col=}\StringTok{'red'}\NormalTok{, }\DataTypeTok{add=}\OtherTok{TRUE}\NormalTok{)}
\end{Highlighting}
\end{Shaded}

\includegraphics{sta211_files/figure-latex/unnamed-chunk-24-3.pdf}

\begin{Shaded}
\begin{Highlighting}[]
\KeywordTok{plot}\NormalTok{(}\KeywordTok{mcmc.list}\NormalTok{(}\KeywordTok{list}\NormalTok{(}\KeywordTok{as.mcmc}\NormalTok{(chain1[}\DecValTok{25001}\OperatorTok{:}\DecValTok{100000}\NormalTok{,}\DecValTok{2}\NormalTok{]),}\KeywordTok{as.mcmc}\NormalTok{(chain2[}\DecValTok{25001}\OperatorTok{:}\DecValTok{100000}\NormalTok{,}\DecValTok{2}\NormalTok{]),}\KeywordTok{as.mcmc}\NormalTok{(chain3[}\DecValTok{25001}\OperatorTok{:}\DecValTok{100000}\NormalTok{,}\DecValTok{2}\NormalTok{]))),}\DataTypeTok{trace=}\OtherTok{FALSE}\NormalTok{, }\DataTypeTok{xlab=}\KeywordTok{TeX}\NormalTok{(}\StringTok{'$}\CharTok{\textbackslash{}\textbackslash{}}\StringTok{alpha$'}\NormalTok{),}\DataTypeTok{ylab=}\StringTok{"Density"}\NormalTok{)}
\KeywordTok{curve}\NormalTok{(}\KeywordTok{dgamma}\NormalTok{(x,a,b), }\DataTypeTok{col=}\StringTok{'red'}\NormalTok{, }\DataTypeTok{add=}\OtherTok{TRUE}\NormalTok{)}
\end{Highlighting}
\end{Shaded}

\includegraphics{sta211_files/figure-latex/unnamed-chunk-24-4.pdf}

\begin{Shaded}
\begin{Highlighting}[]
\NormalTok{kappalist=mcmctc[,}\DecValTok{1}\NormalTok{]}
\NormalTok{kappalist=}\KeywordTok{rbind}\NormalTok{(kappalist[[}\DecValTok{1}\NormalTok{]],kappalist[[}\DecValTok{2}\NormalTok{]],kappalist[[}\DecValTok{3}\NormalTok{]])}
\NormalTok{alphalist=mcmctc[,}\DecValTok{2}\NormalTok{]}
\NormalTok{alphalist=}\KeywordTok{rbind}\NormalTok{(alphalist[[}\DecValTok{1}\NormalTok{]],alphalist[[}\DecValTok{2}\NormalTok{]],alphalist[[}\DecValTok{3}\NormalTok{]])}



\KeywordTok{plot}\NormalTok{(}\KeywordTok{as.numeric}\NormalTok{(kappalist),}\KeywordTok{as.numeric}\NormalTok{(alphalist}\OperatorTok{^}\NormalTok{\{}\OperatorTok{-}\DecValTok{1}\OperatorTok{/}\NormalTok{kappalist\}), }\DataTypeTok{col=}\StringTok{'blue'}\NormalTok{, }\DataTypeTok{main=}\StringTok{"Graphe }
\StringTok{des correlations a posteriori"}\NormalTok{, }\DataTypeTok{xlab=}\KeywordTok{TeX}\NormalTok{(}\StringTok{"$}\CharTok{\textbackslash{}\textbackslash{}}\StringTok{kappa$"}\NormalTok{), }\DataTypeTok{ylab=}\KeywordTok{TeX}\NormalTok{(}\StringTok{"$}\CharTok{\textbackslash{}\textbackslash{}}\StringTok{alpha^\{-1/}\CharTok{\textbackslash{}\textbackslash{}}\StringTok{kappa\}$"}\NormalTok{))}
\end{Highlighting}
\end{Shaded}

\includegraphics{sta211_files/figure-latex/unnamed-chunk-25-1.pdf}

On peut remarquer que les résultats renvoyés par les deux algorithmes
sont globalement similaires que ce soit en terme de densité ou en terme
de corrélations des variables. En effet, les moyennes et variances ainsi
que l'aspect global est très semblable. Lorsque l'on compare
l'application des deux algorithmes, le traitement des données nécessaire
aux algorithmes MCMC est conséquent en réflexion et en calcul ce qui est
moins le cas de la première approche bayésienne utilisant le principe
d'acceptation-rejet. Egalement, du point de vue théorique, l'algorithme
MCMC demande plus de réflexion pour être appréhendé. Toutefois, ces
algorithmes ont tout de même des points communs (tests d'acceptation
d'une valeur candidate à chaque itération). De plus, le fait qu'il y n'y
ait pas à faire d'optimisation (contrairement à l'acceptation rejet)
pour trouver de paramètre adéquat, le fait qu'il y ait des résultats de
convergence théorique et enfin le fait qu'il y ait de nombreuses façons
de vérifier ses résultats et la propriété de convergence en pratique
font que l'on préfèrera utiliser l'algorithme MCMC (en plus des
résultats convenables qu'il fournit ici).

\end{document}
